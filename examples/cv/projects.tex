%-------------------------------------------------------------------------------
%	SECTION TITLE
%-------------------------------------------------------------------------------
\cvsection{Projects}


%-------------------------------------------------------------------------------
%	CONTENT
%-------------------------------------------------------------------------------
\begin{cventries}

%---------------------------------------------------------
  \cventry
    {https://github.com/itsmokha/Property-Price-Visualization-Dubai} % Affiliation/role/Github link
    {Dubai Property Market Analysis} % Organization/group/title
    {}
    {}
   %{Erlangen, Deutschland} % Location
   % {Apr. 2019 - Sep. 2019} % Date(s)
    {
      \begin{cvitems} % Description(s) of experience/contributions/knowledge
		\item Analysis of dataset containing over 50,000 listings to visualise property rates in Dubai.
		\item Created the dataset by scraping Bayut.com and currently planning to scrape data from PropertyFinder to increase the dataset for potentially better analysis. 
		\item Identified the overall market rates by neighborhood.
		\item Visualized the price per square foot in Dubai neighborhoods using PowerBI.
      \end{cvitems}
    }

%---------------------------------------------------------
  \cventry
    {https://github.com/itsmokha/Customer-Conversion} % Affiliation/role/Github link
    {Customer Conversion Rate using Data Analysis} % Organization/group/title
    {}
    {}
    %{Erlangen, Deutschland} % Location
    %{Mar. 2018 - Jul. 2018} % Date(s)
    {
      \begin{cvitems} % Description(s) of experience/contributions/knowledge
		\item Analysis of a dataset containing 67000+ rows of data to analyze the effect of promotional campaigns on customer conversion rate  and graphs to provide visual representation of collected data and gain insights on customer behavior.
		\item Achieved sentiment classification accuracy of 86\% using various various supervised Machine Learning algorithms such as neural networks, regression, XGBoost and Random forests, logistic regression.
		\item Identified the best responding group for promotions using Uplift Modeling and used Qini curves to visualize accuracy.
      \end{cvitems}
    }

%---------------------------------------------------------
  \cventry
    {https://github.com/itsmokha/Advertisement-detection-using-machine-learning} % Affiliation/role/Github link
    {Advertisement Detection using Machine Learning} % Organization/group/title
    {}
    {}
    %{Erlangen, Deutschland} % Location
    %{Apr. 2014 - Dez. 2014} % Date(s)
    {
      \begin{cvitems} % Description(s) of experience/contributions/knowledge
		\item Final year project where URLs were scanned to detect advertisement links using libraries such as Scikit-Learn, Tensorflow, Pandas, and Seaborn.
		\item Created a dataset using various sources totaling more than a million elements and extracted necessary features from the data.
		\item Used natural language processing to extract identifying features from links.
		\item Obtained an accuracy of 80\% using various supervised Machine Learning algorithms such as neural networks, regression, XGBoost, Support Vector Machines, LightGBM and Random forests and performed hyperparameter tuning to find the optimal parameters for the model.
      \end{cvitems}
    }

%---------------------------------------------------------
  \cventry
    {Scikit-Learn, Tensorflow, Seaborn} % Affiliation/role/tools used
    {Amazon Review Sentiment Analysis} % Organization/group
    {}
    {}
    %{Erlangen, Deutschland} % Location
    %{Mar. 2013 - Jul. 2013} % Date(s)
    {
      \begin{cvitems} % Description(s) of experience/contributions/knowledge
		\item Analysis of dataset containing over 227,000 Amazon reviews to identify customer satisfaction with the product  with the help of natural language processing, using Tensorflow and graphs to provide visual representation of the data.
		\item Achieved sentiment classification of 90\% using supervised Machine Learning algorithms such as neural networks, regression, XGBoost , LightGBM, Random forests, SVM (Support Vector Machines) classifiers and hyper parameter tuning.
		\item Identified overall customer satisfaction with various products and made inferences on best selling products.
		\item Visualized the most common sentiments regarding various products within the data.
      \end{cvitems}
    }

%---------------------------------------------------------
\cventry
{Scikit-Learn, Tensorflow, Seaborn} % Affiliation/role/tools used
{Twitter Sentiment Analysis} % Organization/group
{}
{}
%{Erlangen, Deutschland} % Location
%{Mar. 2013 - Jul. 2013} % Date(s)
{
	\begin{cvitems} % Description(s) of experience/contributions/knowledge
	\item Analysis of dataset containing over 10,000 tweets to identify the sentiment expressed with the help of natural language processing, using Tensorflow and graphs to provide visual representation of collected data.
	\item Scraped tweets from Twitter using Python, and extracted, transformed and loaded (ETL) them to gain valuable insights.
	\item Achieved sentiment classification accuracy of 92\% using various supervised Machine Learning algorithms such as neural networks, logistic regression, XGBoost and Random forests.
	\item Identified common sentiments regarding artificial intelligence, and visualized the common themes within the data using word clouds and other graphs.
	\end{cvitems}
}

%---------------------------------------------------------
\cventry
{Jupyter Notebook, Scikit-Learn, Tensorflow, Seaborn} % Affiliation/role/tools used
{Airline Passenger Satisfaction} % Organization/group
{}
{}
%{Erlangen, Deutschland} % Location
%{Mar. 2013 - Jul. 2013} % Date(s)
{
	\begin{cvitems} % Description(s) of experience/contributions/knowledge
		\item Analysis of customer reviews to analyze customer satisfaction with airline.
		\item Thorough data analysis and machine learning models were conducted on airline customer review dataset.
		\item Worked with a mix of categorical as well as numerical data.
		\item Provided visual representation of the results using various graphs.
        \item Used unsupervised and supervised machine learning algorithms such as Clustering, Decision Trees, K nearest neighbors and various others to analyse overall satisfaction.
		%\item Applied concepts such as Clustering, Decision Trees, K nearest neighbors and various others.		
	\end{cvitems}
}

%---------------------------------------------------------
\cventry
{https://github.com/HWTechClub/Virus- Scanner} % Affiliation/role/tools used
{vscan} % Organization/group
{}
{}
%{Erlangen, Deutschland} % Location
%{Mar. 2013 - Jul. 2013} % Date(s)
{
	\begin{cvitems} % Description(s) of experience/contributions/knowledge
		\item A Python-based virus scanner that uses different APIs to scan files and websites for malicious content.
		\item Worked as project lead and identified the best tools to use.
		\item Improved security by implementing best practices regarding API keys
		\item Published the project on PyPi, making it pip-installable.
	\end{cvitems}
}
%---------------------------------------------------------
\end{cventries}
